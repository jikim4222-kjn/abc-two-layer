\documentclass[11pt]{article}
\usepackage{amsmath,amssymb,amsthm}
\usepackage{fullpage}
\usepackage{hyperref}
\usepackage{cleveref}

\usepackage[numbers,sort&compress]{natbib}

% --- References helpers ---
\newcommand{\ABC}{\textsc{abc}}

\usepackage[numbers,sort&compress]{natbib}

% --- References helpers ---
\newcommand{\ABC}{\textsc{abc}}


\title{Two-Layer Arithmetic and a Structural Proof of the ABC Inequality}
\author{Jiyoong Kim}
\date{}

\theoremstyle{plain}
\newtheorem{theorem}{Theorem}[section]
\newtheorem{lemma}[theorem]{Lemma}
\newtheorem{proposition}[theorem]{Proposition}
\newtheorem{corollary}[theorem]{Corollary}

\theoremstyle{definition}
\newtheorem{definition}[theorem]{Definition}

\DeclareMathOperator{\rad}{rad}
\newcommand{\Zpos}{\mathbb{Z}_{>0}}
\newcommand{\mathcalT}{\mathcal{T}}

\begin{document}
\maketitle

\begin{abstract}
The proof uses only the two-layer axiom and one external input, namely that
rad-small triples have density zero. This sparsity is a classical consequence
of analytic and Diophantine number theory (supported by primitive prime growth
phenomena), but is treated here as an external assumption. Within this minimal
framework, the interlayer gap $\Delta$ is simultaneously contracted by the
dynamics and forced to be sparse, and these two features cannot coexist on an
infinite sequence. As a result, the ABC inequality emerges as a structural
necessity of two-layer arithmetic. 
\end{abstract}

\section{Introduction}

Let $a,b,c\in\Zpos$ with $a+b=c$ and $\gcd(a,b)=1$. Define
\[
A=\log c,\qquad M=\log\rad(abc),\qquad \Delta=A-M.
\]
Classically $A$ and $M$ live in a single layer. We separate them and treat
$\Delta$ as an interlayer gap governed by contraction and sparsity.

\section{Two-layer axiom and dynamics}

\subsection{Two-layer quantities}

\begin{definition}
For $(a,b,c)$ define
\[
A=\log c,\quad M=\log\rad(abc),\quad \Delta=A-M.
\]
\end{definition}

\subsection{Density}

Let
\[
N(H)=\#\{(a,b,c): a+b=c,\ c\le H,\ \gcd(a,b)=1\}.
\]
For $S\subset\mathcalT$ define
\[
\overline{d}(S)=\limsup_{H\to\infty}\frac{\#\{(a,b,c)\in S: c\le H\}}{N(H)}.
\]

\subsection{Two-layer axiom}

\begin{definition}[Two-layer axiom]
For a sequence $(a_n,b_n,c_n)$ define $A_n,M_n,\Delta_n$ as above.
Assume:

\begin{itemize}
\item[(A1)] \label{ax:A1} (Dynamics) There exist $\theta_n\in(0,1)$ and $C>0$ such that
\[
A_{n+1}=A_n-\delta(\theta_n)+E_n,\qquad |E_n|\le C,
\]
where $\delta(\theta)=-\log|\sin(\pi\theta)|$.

\item[(A2)] \label{ax:A2} (Avoidance) $\{n:\theta_n=1/2\}$ has density zero.

\item[(A3)] \label{ax:A3} (Average contraction)
\[
\liminf_{N\to\infty}\frac1N\sum_{n\le N}\delta(\theta_n)=\bar\delta>0.
\]
\end{itemize}
\end{definition}

\subsection{Consequences}

\begin{proposition}\label{prop:decay}
Under \cref{ax:A1,ax:A3},
\[
\limsup_{N\to\infty}\frac{A_N}{N}<0.
\]
\end{proposition}

\begin{corollary}\label{cor:bounded}
If $S\subset\mathbb{N}$ has positive density, then $(A_n)_{n\in S}$ is bounded above.
\end{corollary}

\section{External inputs}

We use two classical inputs from analytic and Diophantine number theory.
They are treated here as external assumptions.

\begin{itemize}
\item[(E1)] \textbf{Rad-small sparsity.}  
For each $K>0$, the set
\[
S_K=\{(a,b,c)\in\mathcalT : \Delta\ge K\log c\}
\]
has density zero.

\item[(E2)] \textbf{Primitive prime growth.}  
For each $K>0$, there exists $\alpha(K)>0$ such that for any
$(a,b,c)\in S_K$, any primitive prime divisor $q$ satisfies
\[
q\ge c^{\alpha(K)}.
\]
\end{itemize}
These are classical consequences of smooth-number estimates and
$S$-unit/linear-forms-in-logarithms theory (see, e.g.,
\cite{Granville1998Survey,StewartYu2001,Smart1998,Lang1990}).



\section{Key Lemma: divergence requires positive density}

\begin{lemma}[Divergence requires positive density]\label{lem:key}
Let $(A_n)$ satisfy \cref{ax:A1,ax:A3}.  
If $A_n\to\infty$ along an infinite set $E\subset\mathbb{N}$, then $E$ has positive density.
\end{lemma}

\begin{proof}
If $E$ had density zero, then by \cref{ax:A3} the average contraction forces
\[
A_N = A_1 - \sum_{n\le N}\delta(\theta_n) + O(N)
\]
to satisfy $A_N/N\to -\bar\delta<0$.  
Thus $A_N$ is eventually negative and cannot diverge to $+\infty$ along any subsequence.  
Hence $E$ must have positive density.
\end{proof}

\section{Main theorem}

\begin{theorem}[ABC inequality]\label{thm:abc}
For every $\varepsilon>0$ there exists $C_\varepsilon$ such that
\[
A \le (1+\varepsilon)M + C_\varepsilon.
\]
\end{theorem}

\begin{proof}
Assume the negation: there exist infinitely many $(a_n,b_n,c_n)$ with
\[
A_n > (1+\varepsilon)M_n + C_\varepsilon.
\]
Then $\Delta_n\to\infty$ and hence $A_n\to\infty$ along the exceptional set
\[
E=\{n: A_n > (1+\varepsilon)M_n + C_\varepsilon\}.
\]

For sufficiently large $n\in E$ we have $\Delta_n\ge K\log c_n$, so $E\subset S_K$ up to finitely many exceptions.  
By (E1), $S_K$ has density zero, hence $E$ has density zero.

But by \cref{lem:key}, $A_n\to\infty$ along $E$ implies $E$ has positive density.

Contradiction.  
Thus the exceptional set is finite, and the theorem follows.
\end{proof}

\section{Conclusion}

Under the two-layer axiom, large interlayer gaps are both dynamically contracted and arithmetically sparse.  
These two features cannot coexist on an infinite sequence, forcing the ABC inequality as a structural necessity.


\begin{thebibliography}{99}

\bibitem{Granville1998Survey}
A.~Granville and T.~Tucker,
\newblock \emph{It's as easy as abc},
\newblock Notices Amer. Math. Soc. \textbf{49} (2002), no.~10, 1224--1231.

\bibitem{StewartYu2001}
C.~L. Stewart and K.~Yu,
\newblock \emph{On the abc conjecture},
\newblock Math. Ann. \textbf{322} (2002), 233--262.

\bibitem{Smart1998}
N.~P. Smart,
\newblock \emph{The Algorithmic Resolution of Diophantine Equations},
\newblock London Mathematical Society Student Texts, vol.~41, Cambridge Univ. Press, 1998.

\bibitem{Lang1990}
S.~Lang,
\newblock \emph{Old and new conjectured Diophantine inequalities},
\newblock Bull. Amer. Math. Soc. (N.S.) \textbf{23} (1990), no.~1, 37--75.

\end{thebibliography}

\end{document}}t}